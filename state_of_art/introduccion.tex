\pagebreak
\section{Introducción}

En esta nueva era digital en la que estamos viviendo, la tecnología ha tomado un
rumbo en donde los dispositivos electrónicos y eléctricos se han vuelto
indispensables en la vida diaria ya sea para responder a un jefe, acordar un
evento con unos amigos, entretenerse etc. Y todo esto se considera como una
generación que comenzó desde computadoras de escritorio hasta teléfonos
inteligentes que al menos nacionalmente son mayores que la población total:
``Por citar un simple dato, en el año 2021 en Colombia la población era cercana
a los 47 millones de habitantes y existían aproximadamente 60 millones de
aparatos celulares activos.'' (MinTIC, 2021  como se cita en \cite{Trejos2023})

Estos aparatos otorgan la posibilidad de cumplir con nuestras tareas en los
contextos actuales y ofrecen una cantidad masiva de información. Sin embargo,
detrás de dichas comodidades, se esconden desafíos importantes que lideran y
amenazan el mantener un uso correcto del funcionamiento de estos dispositivos y
la seguridad de nuestros datos. Tales desafíos se caracterizan con encontrar
fallas en los dispositivos que comúnmente se tienen en nuestras casas y que por
norma general al ser bastante complejos, se tiene poco conocimiento de lo que
ocurre tanto a nivel físico (hardware) como a nivel lógico (software) y solo
disponemos de lo que se ve en las pantallas. 

Mantener un computador completamente funcional resulta bastante problemático al
componerse de circuitos eléctricos que con el paso del tiempo pueden
deteriorarse y en el proceso se generan alteraciones muy molestas, como
``artefactos'' en el monitor, que la máquina se apague durante la realización de
una actividad, que de repente el sonido no funcione, que por ninguno de los
medios se puede tener conexión a internet etc.

Estas alteraciones no solo causan inconvenientes y frustración para los
usuarios, sino también alimentan problemas como la obsolescencia tecnológica y
las fallas de hardware. En donde se examina que los dispositivos son desechados
debido a que sus componentes dejan de ser compatibles, no tienen el rendimiento
que se tenía desde su fabricación o simplemente dejan de funcionar por un
problema físico del que los usuarios tienen control nulo. Estas situaciones
obligan a los usuarios a tener que hacer cambios en la máquina que se tiene,
pero aquello representa un gasto significativo para una organización que cuente
con varios dispositivos, y ni hablar de los usuarios domésticos.

Este ciclo constante de reemplazo surge de la necesidad de una solución integral
que aborde tanto el mantenimiento preventivo de hardware y software de la mano
con la ciberseguridad. Siendo esta última una cuestión que resulta preocupante,
pues no solo deja comprometido el estado de los archivos de un equipo, sino que
también puede influir en otros aspectos como la economía, movimientos sociales,
aspectos ecológicos o la integridad total de una maquinaria pensada para operar
con elementos masivos, como las fábricas.

Ante estas dos cuestiones y sumado a la necesidad del mundo actual de tener
conocimiento de las cosas que hay alrededor, surgió la idea de crear una
aplicación que pueda otorgar conocimiento de los elementos que son problemáticos
tanto en el hardware, como de la seguridad de la información de un dispositivo
de cómputo, siendo esta última una cuestión que cada vez evoluciona de acuerdo a
los problemas que surgen ante el control de la información que se desea
almacenar.
