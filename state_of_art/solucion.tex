\pagebreak
\section{Solución}

Normalmente, cuando se piensa en el tema del malware, de la seguridad, de la
optimización del rendimiento de un computador y asegurar una garantí mayor,
directamente se asume como solución el uso de un antivirus. En el caso de las
PyMES se le incorporaría la acción de concientización a partir de conferencias
para hacer una gestión de todo lo que se está haciendo y como se refiere
\textcite{Bueno2022}, aplicar una décima parte del presupuesto. Y en el caso de
los usuarios domésticos, estos deben tener cierta cultura de los elementos que
tiene su máquina, para saber las acciones inmediatas ante un problema menor y
aplicarlas de forma sabia sin necesidad de gastar dinero en reparaciones junto
con licencias a las que se les debe pagar en un cirtos periodos de tiempo para
su uso.

Por lo tanto, se pensó en la idea de crear un software inspirado en los
programas antivirus y con Ada Test incorporado para prevenir fallos
catastróficos sin el conocimiento de los usuarios. Estas herramientas con las
características dadas por \textcite{Chen2023}, \textcite{Ortiz2024} y
\textcite{Zailani2021}, se pueden utilizar para crear un software ayudar a
identificar elementos que hacen ligeras perturbaciones en los archivos y que
conllevan a que se puedan generar vulnerabilidades que generen los efectos
explicados por \textcite{Mosquera2019} y \textcite{Rosli2019}. Este software a
su vez se deja disposición una versión gratuita y una de pago, sin necesidad de
recurrir a la atención en anuncios en plena ejecución de la aplicación, que
como comenta \textcite{Hartl2023} es molesto. 

La versión de pago estará dedicada a funcionalidades más sofisticadas que las
que ofrece el software gratuito, es decir aquellas que \textcite{Singh2022} y
\textcite{Waqar2023} comentan para dispositivos de internet de las cosas.
Esto es así, puesto que aunque el IoT cada vez se hace más presente, en el uso
doméstico es poco común, bastante costoso y que puede tener mayor influencia
de problemas de brechas de seguridad expuestas por \textcite{Rosli2019}. Y como
el Deep Learning requiere de una maquinaria pesada, es necesario contar con
base monetaria para mantenerla y seguir puliendo la detección de malware.

Esta noción fue inspirada por los elementos presentados por
\textcite{Aboaoja2022} al intentar realizar un análisis con el malware que
evoluciona con el paso del tiempo y que influye en aspectos que indirectamente
están relacionados con la informática, y ni hablar lo que ocurre a nivel
nacional según \textcite{CanoMartinez2022}.

Para crear el software de una forma para los usuarios que cuentan con un
inadecuado funcionamiento de un equipo de cómputo, se tiene que disponer de un
medio intuitivo que cumpla con las características otorgadas por
\textcite{Mena2022} de tal forma que cumpla no solo con la calidad del software
ante los usuarios, sino también para que sea útil para aquellos que usan el
software a desarrollar del mantenimiento preventivo:

\begin{displayquote}
  La usabilidad pasó a ser un factor para medir la calidad de un producto de
  software, pues se debe garantizar la eficiencia, eficacia, satisfacción en el
  uso del producto y su relación con los usuarios; por lo que llega a ser
  considerada fuertemente en el área del diseño de interfaces y la interacción
  humano computador, para buscar la facilidad de uso de un producto.
\end{displayquote}

Y en lo que se revisó de \textcite{Trejos2023}, se concluyó que el uso del
lenguaje de programacón Java sería útil para la creación de una interfaz que
caracteriza su usabilidad, y el uso del lenguaje de programación C para tener
control tanto de los elementos de bajo en la perspectiva de alto nivel. Y 
sabiendo que tales lenguajes representan una oportunidad para usar Ada Test para
beneficiar a los usuarios que cuenten con un equipo que funciona de forma
deficiente.
