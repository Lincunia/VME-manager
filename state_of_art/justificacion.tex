\pagebreak
\section{Justificación}

Pese a lo que \textcite{Herat2020} detalla sobre las soluciones proporcionadas
por las empresas, \textcite{Hartl2023} encontraron que no todas las personas son
conscientes de este hecho, y por ende sus productos se añaden a los valores de
los desperdicios eléctricos y electrónicos que \textcite{Seif2024} menciona de
forma general. Considerando que en la mayoría de los casos se asegura un periodo
de tiempo para mantener a flote la economía de las firmas, al ver que los
usuarios tienen el deseo de comprar un aparato electrónico de mayor calidad una
vez se presentan fallos. Sin embargo, no todos se encuentran con la posibilidad
de reemplazar una máquina en periodos de tiempo, en particular si estos son
cortos, cosa que resulta costosa para los usuarios.

Y no solo en los fallos físicos por desgaste o los fallos internos inesperados,
sino que como se vieron con las revisiones hechas por \textcite{Rathore2023} o
\textcite{Waqar2023} (En el caso de los dispositivos móviles) sobre el tema de
la ciberseguridad, a los que \textcite{CanoMartinez2022} considera como
primordiales para los aspectos nacionales más importantes como la economía o la
estabilidad de las empresas. Asumiendo que 53.47\% de los usuarios
mayores son engañados por mensajes que ofrecen algo, se consideró que se tiene
que hacer un software que cumpla las características de prevenir al usuario de
malware, además de monitorear y analizar el rendimiento del equipo para su
mantenimiento preventivo. Todo esto junto con una guía del lenguaje que se tiene
sobre los sistemas informáticos actuales de tal modo que, como lo sugiere
\textcite{Nurul2021}, se pueda tener una idea sobre cómo las aplicaciones y el
software a desarrollar funcionan en los equipos de cómputo. 

Por lo tanto, la idea de la aplicación a desarrollar se basa principalmente en
reducir los riesgos que se generan, y según lo expresa \textcite{Mosquera2019},
al realizar varias de las actividades que involucran directamente el
desconocimiento de la información que se suele presentar por parte los nuevos
usuarios, muchos atacantes aprovechan los vacíos de seguridad que los presenta
y por ende, hacerlos quedar expuestos a varias amenazas que atentan con la
privacidad, la identidad, el estado de una organización o las infraestructuras
críticas. \parencite[e.g.,][]{CanoMartinez2022} 
