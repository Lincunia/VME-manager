\pagebreak
\section{Pregunta generadora}

En la búsqueda de información sobre nuestra problemática, se propuso la creación
de un software que se dedique a otorgarle a un usuario la mejor acción que puede
realizar para prevenir fallos de su computadora para que pueda realizar las
acciones que necesita en su día a día. Pero en esas circunstancias, se planteó
una pregunta que resulta ser el común denominador de los campos que involucra el
software: ¿Qué elementos hay que tener en cuenta para desarrollar el software?

Esta pregunta lo que lleva a un desarrollador es a indagar sobre los conceptos
en lo que es sumamente fatuo para posteriormente, tener un esquema lo
suficientemente satisfactorio en el sentido de que se pueda dar a conocer las
respuestas ante las incógnitas que se genera: Como el qué hacer para crear un
sistema inteligente sea capaz de distinguir entre acciones bajo la condición del
usuario y las de agentes externos, o cómo comparar rendimientos en diferentes
periodos de acuerdo al uso y qué consejos puede dar al usuario de tal modo que
no se encuentre con problemas e incógnitas a la hora de solucionar tal
situación.

Y una de las preguntas que formuló \textcite{Trejos2023} para conocer cuál es la
mejor manera de aprender a programar en entornos académicos, como lo son el de
una universidad pública: ``¿Qué favorece más en el aprendizaje de la
programación: los fundamentos matemáticos y lógicos o los lenguajes de
programación?'' De esta pregunta nos lleva a pensar en el contexto de realizar
un mantenimiento preventivo en el hardware y software como el preguntarnos sobre
cuáles son los medios indicados para dar los resultados que se espera hacia los
usuarios. Y sobre todo, para hacerlos lo suficientemente rápidos como para
satisfacer las necesidades actuales.
